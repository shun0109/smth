\documentclass[fleqn]{article}
\setlength{\mathindent}{3pt}

%%%%%%%%%%%%%%%%%%%%%%%%%%%%%%%%%%%%%%%%%
% Lachaise Assignment
% Structure Specification File
% Version 1.0 (26/6/2018)
%
% This template originates from:
% http://www.LaTeXTemplates.com
%
% Authors:
% Marion Lachaise & François Févotte
% Vel (vel@LaTeXTemplates.com)
%
% License:
% CC BY-NC-SA 3.0 (http://creativecommons.org/licenses/by-nc-sa/3.0/)
% 
%%%%%%%%%%%%%%%%%%%%%%%%%%%%%%%%%%%%%%%%%

%----------------------------------------------------------------------------------------
%	PACKAGES AND OTHER DOCUMENT CONFIGURATIONS
%----------------------------------------------------------------------------------------

\usepackage{amsmath,amsfonts,stmaryrd,amssymb} % Math packages

\usepackage{enumerate} % Custom item numbers for enumerations

\usepackage[ruled]{algorithm2e} % Algorithms

\usepackage[framemethod=tikz]{mdframed} % Allows defining custom boxed/framed environments

\usepackage{listings} % File listings, with syntax highlighting
\lstset{
	basicstyle=\ttfamily, % Typeset listings in monospace font
}

%----------------------------------------------------------------------------------------
%	DOCUMENT MARGINS
%----------------------------------------------------------------------------------------

\usepackage{geometry} % Required for adjusting page dimensions and margins

\geometry{
	paper=a4paper, % Paper size, change to letterpaper for US letter size
	top=2.5cm, % Top margin
	bottom=3cm, % Bottom margin
	left=2.5cm, % Left margin
	right=2.5cm, % Right margin
	headheight=14pt, % Header height
	footskip=1.5cm, % Space from the bottom margin to the baseline of the footer
	headsep=1.2cm, % Space from the top margin to the baseline of the header
	%showframe, % Uncomment to show how the type block is set on the page
}

%----------------------------------------------------------------------------------------
%	FONTS
%----------------------------------------------------------------------------------------

\usepackage[utf8]{inputenc} % Required for inputting international characters
\usepackage[T1]{fontenc} % Output font encoding for international characters

\usepackage{XCharter} % Use the XCharter fonts

%----------------------------------------------------------------------------------------
%	COMMAND LINE ENVIRONMENT
%----------------------------------------------------------------------------------------

% Usage:
% \begin{commandline}
%	\begin{verbatim}
%		$ ls
%		
%		Applications	Desktop	...
%	\end{verbatim}
% \end{commandline}

\mdfdefinestyle{commandline}{
	leftmargin=10pt,
	rightmargin=10pt,
	innerleftmargin=15pt,
	middlelinecolor=black!50!white,
	middlelinewidth=2pt,
	frametitlerule=false,
	backgroundcolor=black!5!white,
	frametitle={Command Line},
	frametitlefont={\normalfont\sffamily\color{white}\hspace{-1em}},
	frametitlebackgroundcolor=black!50!white,
	nobreak,
}

% Define a custom environment for command-line snapshots
\newenvironment{commandline}{
	\medskip
	\begin{mdframed}[style=commandline]
}{
	\end{mdframed}
	\medskip
}

%----------------------------------------------------------------------------------------
%	FILE CONTENTS ENVIRONMENT
%----------------------------------------------------------------------------------------

% Usage:
% \begin{file}[optional filename, defaults to "File"]
%	File contents, for example, with a listings environment
% \end{file}

\mdfdefinestyle{file}{
	innertopmargin=1.6\baselineskip,
	innerbottommargin=0.8\baselineskip,
	topline=false, bottomline=false,
	leftline=false, rightline=false,
	leftmargin=2cm,
	rightmargin=2cm,
	singleextra={%
		\draw[fill=black!10!white](P)++(0,-1.2em)rectangle(P-|O);
		\node[anchor=north west]
		at(P-|O){\ttfamily\mdfilename};
		%
		\def\l{3em}
		\draw(O-|P)++(-\l,0)--++(\l,\l)--(P)--(P-|O)--(O)--cycle;
		\draw(O-|P)++(-\l,0)--++(0,\l)--++(\l,0);
	},
	nobreak,
}

% Define a custom environment for file contents
\newenvironment{file}[1][File]{ % Set the default filename to "File"
	\medskip
	\newcommand{\mdfilename}{#1}
	\begin{mdframed}[style=file]
}{
	\end{mdframed}
	\medskip
}

%----------------------------------------------------------------------------------------
%	NUMBERED QUESTIONS ENVIRONMENT
%----------------------------------------------------------------------------------------

% Usage:
% \begin{question}[optional title]
%	Question contents
% \end{question}

\mdfdefinestyle{question}{
	innertopmargin=1.2\baselineskip,
	innerbottommargin=0.8\baselineskip,
	roundcorner=5pt,
	nobreak,
	singleextra={%
		\draw(P-|O)node[xshift=1em,anchor=west,fill=white,draw,rounded corners=5pt]{%
		Question \theQuestion\questionTitle};
	},
}

\newcounter{Question} % Stores the current question number that gets iterated with each new question

% Define a custom environment for numbered questions
\newenvironment{question}[1][\unskip]{
	\bigskip
	\stepcounter{Question}
	\newcommand{\questionTitle}{~#1}
	\begin{mdframed}[style=question]
}{
	\end{mdframed}
	\medskip
}

%----------------------------------------------------------------------------------------
%	WARNING TEXT ENVIRONMENT
%----------------------------------------------------------------------------------------

% Usage:
% \begin{warn}[optional title, defaults to "Warning:"]
%	Contents
% \end{warn}

\mdfdefinestyle{warning}{
	topline=false, bottomline=false,
	leftline=false, rightline=false,
	nobreak,
	singleextra={%
		\draw(P-|O)++(-0.5em,0)node(tmp1){};
		\draw(P-|O)++(0.5em,0)node(tmp2){};
		\fill[black,rotate around={45:(P-|O)}](tmp1)rectangle(tmp2);
		\node at(P-|O){\color{white}\scriptsize\bf !};
		\draw[very thick](P-|O)++(0,-1em)--(O);%--(O-|P);
	}
}

% Define a custom environment for warning text
\newenvironment{warn}[1][Warning:]{ % Set the default warning to "Warning:"
	\medskip
	\begin{mdframed}[style=warning]
		\noindent{\textbf{#1}}
}{
	\end{mdframed}
}

%----------------------------------------------------------------------------------------
%	INFORMATION ENVIRONMENT
%----------------------------------------------------------------------------------------

% Usage:
% \begin{info}[optional title, defaults to "Info:"]
% 	contents
% 	\end{info}

\mdfdefinestyle{info}{%
	topline=false, bottomline=false,
	leftline=false, rightline=false,
	nobreak,
	singleextra={%
		\fill[black](P-|O)circle[radius=0.4em];
		\node at(P-|O){\color{white}\scriptsize\bf i};
		\draw[very thick](P-|O)++(0,-0.8em)--(O);%--(O-|P);
	}
}

% Define a custom environment for information
\newenvironment{info}[1][Info:]{ % Set the default title to "Info:"
	\medskip
	\begin{mdframed}[style=info]
		\noindent{\textbf{#1}}
}{
	\end{mdframed}
}


\title{Heat Kernel Expansion}

\begin{document}

\maketitle
\section{Method}
\subsection{Effective Action}
Let us start from the computation of the effective action in general. Given the classical Lagrangian, as we have seen it can be rewritten in terms of the renormalized field $\phi _r$:
\begin{align}
\mathcal{L} [\phi_r] = \mathcal{L}_r[\phi_r] + \Delta \mathcal{L}[\phi_r]
\end{align}
Then, introduce the external source $J$ and also split it as:
\begin{align}
J(x) = J_r(x) + \Delta J(x)
\end{align}
where, with the expansion $\phi_r(x) = \phi_{cl} (x) + \eta (x)$:
\begin{align}
&\left. \frac{\delta S_r [\phi _r]}{\delta \phi _r (x)} \right| _{\phi_r = \phi_{cl}} = -J_r(x) \\
&\phi_{cl} = \left. \langle \phi(x) \rangle \right|_{J_r + \Delta J}
\end{align}
To proceed, one may consider the functional $Z[J]$ in the following form:
\begin{align}
Z[J] &= \int \mathcal{D}\phi \text{exp}\lbrace i\int d^{d}x (\mathcal{L}_{r} [\phi_r] + J(x) \phi_{r}(x)) \rbrace \nonumber \\
&= \int \mathcal{D} \phi \text{exp} \lbrace i\int d^{d}x (\mathcal{L}_r[\phi_r]  + J_{r}(x)\phi_{r}(x) + \Delta \mathcal{L}[\phi_r] + \Delta J(x) \phi_{r}(x)) \rbrace
\end{align}
The action can be expanded around the background $\phi_{cl}$. The first two terms are:
\begin{align}
\int d^{d}x (\mathcal{L}_r[\phi_r]  + J_{r}(x)\phi_{r}(x))  = &\int d^{d}x (\mathcal{L}_r[\phi_{cl}]  + J_{r}(x)\phi_{cl}(x)) \nonumber \\
&+ \int d^{d}x \eta(x) (\left.\frac{\delta S_{r}[\phi_{r}]}{\delta \phi_{r}(x)} \right|_{\phi_{r} = \phi_{cl}} + J_{r}(x) ) \nonumber  \\
& + \frac{1}{2} \int d^{d}x d^{d}y \eta (x)\eta (y) \left. \frac{\delta^{2} S_{r}[\phi_{r}]}{\delta \phi_{r}(x) \delta \phi_{r} (y)} \right|_{\phi_{r} = \phi_{cl}} \nonumber \\
&+ \cdots
\end{align}
The second term vanishes by the classical equation of motion. The last two terms of the action is also expanded:
\begin{align}
\int d^{d}x (\Delta \mathcal{L}_r[\phi_r]  + \Delta J_{r}(x)\phi_{r}(x))  = &\int d^{d}x (\Delta \mathcal{L}_r[\phi_{cl}]  + \Delta J_{r}(x)\phi_{cl}(x)) \nonumber \\
&+ \int d^{d}x \eta(x) (\left.\frac{\delta \Delta S_{r}[\phi_{r}]}{\delta \phi_{r}(x)} \right|_{\phi_{r} = \phi_{cl}} + \Delta J_{r}(x) ) \nonumber  \\
& + \frac{1}{2} \int d^{d}x d^{d}y \eta (x)\eta (y) \left. \frac{\delta^{2} \Delta S_{r}[\phi_{r}]}{\delta \phi_{r}(x) \delta \phi_{r} (y)} \right|_{\phi_{r} = \phi_{cl}} \nonumber \\
&+ \cdots
\end{align}
The second term stands for a tadpole and must be canceled in such a way that $\langle \eta(x) \rangle _{J} = 0$, i.e, $\langle \phi _{r}\rangle _{J} = \phi _{cl}$. The other terms represent as the counter-terms for the self-interaction vertices. In total:
\begin{align}
Z[J] = &\text{exp} \lbrace i\int d^{d}x  (\mathcal{L}_r[\phi_{cl}]  + J_{r}(x)\phi_{cl}(x) +\Delta \mathcal{L}_r[\phi_{cl}]  + \Delta J_{r}(x)\phi_{cl}(x) )\rbrace   \nonumber \\
& \int \mathcal{D} \eta \text{exp} \lbrace  i\tilde{S}[\eta] + i\Delta \tilde{S}[\eta]  \rbrace
\end{align}
where:
\begin{align}
\tilde{S}[\eta] & = \frac{1}{2} \int d^{d}x d^{d}y  \eta (x) (\frac{\delta ^{2} S_{r}}{\delta \phi_{r}^{2}}[\phi_{cl}](x,y))\eta(y) + \text{vertices.} \\
\Delta \tilde{S}[\eta] & = \text{counter-terms}
\end{align}
Neglecting the interactions, the path integral over $\eta$ is of Gaussian form and can be integrated explicitly. The generating functionals $Z[J]$ takes the form $Z[J] = \text{exp} [iW[J]]$, and $W[J]$ can be written in:
\begin{align}
W[J] = &\int d^{d}x \lbrace \mathcal{L}_{r}[\phi _{cl}(x)] + J_{r}(x)\phi_{r}(x) + \Delta \mathcal{L}_{r}[\phi_{cl}] + \Delta J_{r}(x) \phi _{cl}(x) \rbrace \nonumber \\
& + \frac{1}{2} Tr \ln \frac{\delta ^{2} S_{r}}{\delta \phi_{r}\delta \phi_{r}} [\phi_{cl}] - i(\text{connected diagram})
\end{align}
One may need to perform the Legendre transform  in order to compute the effective action $\Gamma[\phi_{cl}]$:
\begin{align}
\Gamma [\phi_{cl}] &= W[J] - \int d^{d}x J(x)\phi_{cl} (x) \\
&= S_{r}[\phi _{cl}] +\frac{1}{2} Tr \ln \frac{\delta ^{2} S_{r}}{\delta \phi _{r} \delta _{r}}[\phi _{cl}] - i(\text{connected diagram}) + \Delta S[\phi_{cl}]
\end{align}
Our particular interest is in the 1-loop corrections:
\begin{align}
\Gamma^{\text{1-loop}} [\phi_{cl}]  = \frac{1}{2}Tr \ln \frac{\delta ^{2} S_{r}} {\delta \phi_{r} \delta \phi_{r}} + \Delta ^{1} S
\end{align}
So it is necessary to evaluate the trace in some way to know the 1-loop corrections to the effective action.

\subsection{Heat Kernel Expansion}
In order to compute the trace given above, introduce the heat kernel:
\begin{align}
K(t;x,y;D) = \langle x | \text{exp}(-tD) | \rangle
\end{align}
which should satisfy the heat conduction equation:
\begin{align}
(\partial _{t} + D_{x})K(t;x,y;D) = 0
\end{align}
with the initial condition:
\begin{align}
K(0;x,y;D) = \delta (x-y)
\end{align}
For instance, the kernel for $-\Delta$ is:
\begin{align}
K(t;,x,y;-\Delta) = (4\pi t)^{-d/2} \text{exp}(-\frac{(x-y)^{2}}{4t})
\end{align}
and for $D= D_{0} = -\Delta + m^{2}$:
\begin{align}
K(t;x,y;D_{0}) = (4\pi t)^{-d/2} \text{exp}(-\frac{(x-y)^{2}}{4t} - m^{2} t)
\end{align}
For a general D, $K(t;x,y;D_{0})$ still describes the leading singularity as $t \rightarrow 0$ as in the form:
\begin{align}
K(t;x,y;D) = K(t;,x,y;D_{0}) (1 + tb_{2}(x,y) + t^{2} b_{4}(x,y) + \cdots)
\end{align}
The heat kernel coefficients $b_{2k}(x,y)$ are regular n the limit $y \rightarrow x$.\\
Then we need to compute the functional:
\begin{align}
\mathcal{W} = \frac{1}{2} \text{Tr} \ln D
\end{align}
But for each positive eigenvalue $\lambda$  of the operator D, one may have:
\begin{align}
\ln \lambda = - \int_{0}^{\infty} \frac{dt}{t} e^{-t\lambda}
\end{align}
Then,
\begin{align}
\mathcal{W} &=  -\frac{1}{2} \int_{0}^{\infty} \frac{dt}{t} \text{Tr} e^{-tD} \nonumber \\
&= -\frac{1}{2} \int_{0}^{\infty} \frac{dt}{t} \int d^{d}x \sqrt{g} K(t;x,x;D)
\end{align}

\subsection{Generalization}
With those coefficients, consider the functional $\mathcal{W}$, given the quadratic term of the Euclidean action:
\begin{align}
S^{(2)} = \frac{1}{2} \int d^{d}x \Phi^{T} (-\nabla \nabla + Y) \Phi \text{   ,   } \Phi  = \begin{pmatrix}
\phi_{1} \\
\phi_{2} \\
\vdots \\
\phi_{n}
\end{pmatrix}
\end{align}
where $\nabla_{\rho} = \partial _{\rho} + X_{\rho}$, $X_{\rho}^{T} = -X_{\rho}$, and $Y^{T} = Y$.
Then, one may have:
\begin{align}
\mathcal{W} & = -\frac{1}{2} \int _{0}^{\infty} \frac{dt}{t} e^{-tm^{2}}\frac{1}{(4\pi t)^{-d/2}} \int d^{d}x \text{Tr} (1 - tY + \frac{1}{2} t^{2} Y^{2} + \frac{1}{12} t^{2} X_{\mu \nu} X_{\mu \nu} + \cdots ) \nonumber \\
&= \frac{1} {2} \frac{1}{(4\pi)^{d/2}}\int d^{d}x \lbrace m^{d-2} \Gamma(1-\frac{d}{2}) \text{Tr} (Y) - m^{d-4} \Gamma(2-\frac{d}{2}) \text{Tr}(\frac{1}{2}Y^{2} + \frac{1}{12} X_{\mu \nu} X_{\mu \nu}) + \cdots   \rbrace
\end{align}


\section{$\phi^{4}$ case}
Let us apply this method to the $\phi^{4}$ theory, and reproduce its $\beta$ functions. In this case, we have $X = 0$ and $Y = m^{2} + \frac{\lambda}{2} \phi^{2}$. Then:
\begin{align}
\mathcal{W} = \frac{1}{2} \frac{1}{(4\pi)^{d/2}} \int d^{d}x \lbrace \frac{\Gamma(1-\frac{d}{2})}{(m^2)^{{1-\frac{d}{2}}}} (m^{2} + \frac{\lambda}{2}\phi ^{2} )- \frac{\Gamma(2-\frac{d}{2})}{(m^2)^{2-\frac{d}{2}}}\frac{1}{2}(m^{2} + \frac{\lambda}{2}\phi ^{2})^{2} + \cdots \rbrace
\end{align}
Then the counter-terms are:
\begin{align}
\delta _{m^2} &= -\frac{1}{2}\frac{\lambda}{(4\pi)^{d/2} }\frac{\Gamma (1-\frac{d}{2})}{(m^2)^{1-d/2}} \\
\delta _{z} &= 0 \\
\delta _{\lambda} &= \frac{3}{2} \frac{\lambda ^{2}}{(4\pi)^{d/2}} \frac{\Gamma(2-\frac{d}{2})}{(m^2)^{2-d/2}}
\end{align}
Consider the limit $d \rightarrow 4$ with $\epsilon \equiv d -4$, then:
\begin{align}
\delta _{m^{2}} & = -\frac{1}{2}\mu ^{\epsilon} \frac{\lambda}{(4\pi)^{2-\epsilon/2}} \frac{\Gamma(\frac{\epsilon}{2})}{-1+\frac{\epsilon}{2}}\frac{1}{(m^2)^{-1+\epsilon/2}} \nonumber \\
&\overrightarrow{\scalebox{0.7}{$\epsilon \to 0$}} \frac{\lambda}{32\pi ^{2}} m^{2} (\frac{2}{\epsilon} - \gamma _{E} -\ln\frac{m^{2}}{\mu^{2}} + \mathcal{O}(\epsilon)) \nonumber \\
& = \frac{\lambda}{32\pi ^{2}}m^{2} \ln \frac{\zeta}{\mu ^{2}} \\
\delta _{\lambda} & = \frac{3}{2} \frac{\mu ^{\epsilon} \lambda ^{2} }{(4\pi)^{2-\epsilon/2}} \frac{\Gamma(\frac{\epsilon}{2})}{(m^2)^{\epsilon /2}} \nonumber \\
&\overrightarrow{\scalebox{0.7}{$d \to 4$}} \frac{3\lambda^{2}}{32\pi^{2}} \ln \frac{\zeta}{\mu^{2} }
\end{align}
Therefore, $\beta$ functions are:
\begin{align}
\gamma &= 0 \\
\beta & = \frac{3\lambda^{2}}{16\pi ^{2}} \\
\beta_{m^{2}} &= \frac{\lambda}{16\pi^2}
\end{align}


\end{document}
